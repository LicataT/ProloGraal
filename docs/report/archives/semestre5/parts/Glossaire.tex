\documentclass[../report.tex]{subfiles}
\begin{document}
\section{Glossaire}\label{sec:glossary}
\subsection*{ANTLR}
Ensemble de librairies utilisées pour la génération du parser et du lexer.
\subsection*{Deque}
Structure de données supportant l'insertion et la suppression efficace au début et à la fin.
\subsection*{GraalVM}
Plateforme supportant Truffle. Contient une JVM.
\subsection*{Java}
Langage de programmation utilisé pour le projet et ses librairies.
\subsection*{Java Virtual Machine (JVM)}
Machine virtuelle permettant l'exécution de programmes Java.
\subsection*{Maven}
Gestionnaire de dépendances et de cycle de développement.
\subsection*{SimpleLanguage}
Projet de démonstration utilisant Truffle et GraalVM.
\subsection*{Stack (pile)}
Structure de données représentant une pile : le dernier élément inséré est le premier élément qui ressort.
\subsection*{Truffle}
Ensemble de librairies utilisant GraalVM pour fournir des fonctionnalités d'implémentation de langages.
\subsection*{arbre de preuve}
Arbre créé en suivant les règles logiques d'un programme.
\subsection*{arité}
Nombre d'arguments d'une structure Prolog.
\subsection*{backtracking}
Technique de programmation basée sur l'essai systématique de toutes les possiblités.
\subsection*{clauses}
En Prolog, décrit une relation logique. Il existe deux types de clauses : les clauses simples, appelées faits, et les clauses composées.
\begin{itemize}
    \item Un fait est de la forme : \mintinline{Prolog}{fact.}
    \item Une clause composée est de la forme : \mintinline{Prolog}{head :- rule1, rule2.}
\end{itemize}
\subsection*{compilateur à la volée}
Technique permettant de recompiler des bouts de code de manière dynamique lors de l'exécution du programme, pour les rendre plus performants.
\subsection*{contexte}
Contexte de l'application, contenant des informations sur les clauses et sur l'environnement du programme. Dans le projet, le contexte peut désigner deux classes :
\begin{itemize}
    \item La classe \texttt{ProloGraalRuntime}, qui représente le contexte d'exécution et contient les clauses du programme
    \item La classe \texttt{ProloGraalContext}, qui contient le contexte globale avec des références vers la sortie et l'entrée standards du programme
\end{itemize}
\subsection*{faits}
Clause Prolog de la forme : \mintinline{Prolog}{fact.}.
\subsection*{fichier de grammaire}
Fichier contenant du code compréhensible par ANTLR.
\subsection*{foncteur}
Désignateur d'une structure Prolog. Dans la structure \mintinline{Prolog}{a(b, c)}, \texttt{'a'} est le foncteur.
\subsection*{framework}
Ensemble de librairies et autres composants.
\subsection*{interpréteur}
Peut désigner deux choses :
\begin{itemize}
    \item Un interpréteur de langage, qui lit un code source et se charge de l'exécuter (sans le compiler).
    \item L'interpréteur Prolog, qui est une interface en ligne de commande permettant d'effectuer des requêtes sur un programme Prolog.
\end{itemize}
\subsection*{lexer}
Composant chargé de lire un fichier source, et de créer à partir des caractères des unités compréhensibles pour le parser.
\subsection*{librairie}
Ensemble de classes externes réalisant une ou plusieurs fonctionnalités spécifiques.
\subsection*{liste}
En Prolog, désigne une structure de la forme \mintinline{Prolog}{[a,b,c | d]}. Une liste possède possiblement une queue, ou \textit{tail} en anglais, qui désigne l'élément à la fin de celle-ci.
\subsection*{machine virtuelle}
Couche logicielle permettant l'interprétation d'un langage de plus haut niveau.
\subsection*{main}
Fonction principale d'un programme.
\subsection*{open source}
Projet dont le code source est disponible publiquement.
\subsection*{parser}
Composant chargé de donner un sens aux jetons créés par le lexer, afin de créer des structures de plus haut niveau.
\subsection*{pattern de développement}
Patron donnant une réalisation standard d'un composant.
\subsection*{requête}
Question posée à l'interpréteur Prolog.
\subsection*{structure}
En Prolog, structure de forme \mintinline{Prolog}{a(b, c)}.
\subsection*{termes}
En Prolog, mot générique désignant les autres unités de données (structures, variables, etc...).
\subsection*{unification}
Mécanisme permettant de rendre deux termes logiques identiques.
\subsection*{variables}
En Prolog, représente une inconnue logique, capable de prendre une valeur lors de l'unification.
\end{document}