\documentclass[../report.tex]{subfiles}
\begin{document}
\section{Conclusion}
Ce rapport touche à sa fin. Il est temps de faire le point sur le travail réalisé, au travers d'une comparaison avec les objectifs du projet, et de donner un ressenti personnel sur le projet et son déroulement.
\subsection{Atteinte des objectifs}
Les fonctionnalités retenues à la section \ref{subsec:projetobjectives} ont toutes été implémentées avec succès. Les fondations pour certains objectifs optionnels également, aux travers des prédicats intégrés \texttt{var} et \texttt{write}, et une gestion basique des exceptions avec une erreur en cas de prédicat inconnu. Les objectifs ayant été fixés au début du projet sont donc atteints.

Il reste un point noir : le projet s'appelle "Prolog avec Truffle et GraalVM", mais au final, c'est plutôt "Prolog en Java"... Ce n'est cependant pas si grave, vu que le projet laisse de bonnes fondations pour un éventuel approfondissement, et que cette étape était nécessaire pour ensuite intégrer plus profondément Truffle. De plus, l'objectif principal du projet était bien de voir s'il est possible d'implémenter un langage de programmation logique avec Truffle, et nous pouvons maintenant répondre de manière affirmative à cette question.
\subsection{Conclusion personnelle}
J'ai eu du plaisir à réaliser ce projet. J'ai appris beaucoup de choses sur le fonctionnement interne de Prolog et de manière plus générale sur le fonctionnement d'un interpréteur de langage. Le projet s'est bien déroulé, étant resté dans les temps prévus initialement. Je regrette juste un peu de ne pas avoir eu plus de temps pour creuser l'intégration avec Truffle au cours du projet, car il aurait été très intéressant d'essayer d'optimiser l'interpréteur après l'avoir écrit.

Au niveau de la gestion de l'implémentation, il aurait peut-être fallu planifier un peu mieux les différentes phases en pensant un peu plus au futur de chaque fonctionnalité dans le projet final, car je me retrouvais souvant à devoir presque totalement défaire ce qui avait été fait précédemment pour intégrer les nouvelles fonctionnalités (comme on peut le constater particulièrement entre la phase 2 et 3). 

Mais de manière globale, je suis content d'avoir réalisé ce projet et satisfait de son résultat. Je remercie mon superviseur monsieur Frédéric Bapst pour son accompagnement le long de ce projet, ses conseils avisés et son oeil d'aigle pour les fautes de programmation(et d'orthographe !).
\subsection*{Déclaration d'honneur}
\textit{Je, soussigné, Martin Spoto, déclare sur l’honneur que le travail rendu est le fruit d’un travail
personnel. Je certifie ne pas avoir eu recours au plagiat ou à toute autre forme de fraude.
Toutes les sources d’information utilisées et les citations d’auteur ont été clairement
mentionnées.}
\vspace{2cm}
\end{document}